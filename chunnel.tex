% !TEX TS-program = pdflatex
% !TEX encoding = UTF-8 Unicode

% This is a simple template for a LaTeX document using the "article" class.
% See "book", "report", "letter" for other types of document.

\documentclass[12pt]{article} % use larger type; default would be 10pt
\linespread{1.6} % `double' spacing

\usepackage[utf8]{inputenc} % set input encoding (not needed with XeLaTeX)

%%% Examples of Article customizations
% These packages are optional, depending whether you want the features they provide.
% See the LaTeX Companion or other references for full information.

%%% PAGE DIMENSIONS
\usepackage{geometry} % to change the page dimensions
\geometry{a4paper} % or letterpaper (US) or a5paper or....
% \geometry{margin=2in} % for example, change the margins to 2 inches all round
% \geometry{landscape} % set up the page for landscape
%   read geometry.pdf for detailed page layout information

\usepackage{graphicx} % support the \includegraphics command and options

% \usepackage[parfill]{parskip} % Activate to begin paragraphs with an empty line rather than an indent

%%% PACKAGES
\usepackage{booktabs} % for much better looking tables
\usepackage{array} % for better arrays (eg matrices) in maths
\usepackage{paralist} % very flexible & customisable lists (eg. enumerate/itemize, etc.)
\usepackage{verbatim} % adds environment for commenting out blocks of text & for better verbatim
\usepackage{subfig} % make it possible to include more than one captioned figure/table in a single float
% These packages are all incorporated in the memoir class to one degree or another...

%%% HEADERS & FOOTERS
\usepackage{fancyhdr} % This should be set AFTER setting up the page geometry
\pagestyle{fancy} % options: empty , plain , fancy
\renewcommand{\headrulewidth}{0pt} % customise the layout...
\lhead{}\chead{}\rhead{}
\lfoot{}\cfoot{\thepage}\rfoot{}

%%% SECTION TITLE APPEARANCE
\usepackage{sectsty}
\allsectionsfont{\sffamily\mdseries\upshape} % (See the fntguide.pdf for font help)
% (This matches ConTeXt defaults)

%%% ToC (table of contents) APPEARANCE
\usepackage[nottoc,notlof,notlot]{tocbibind} % Put the bibliography in the ToC
\usepackage[titles,subfigure]{tocloft} % Alter the style of the Table of Contents
\renewcommand{\cftsecfont}{\rmfamily\mdseries\upshape}
\renewcommand{\cftsecpagefont}{\rmfamily\mdseries\upshape} % No bold!

%%% END Article customizations

%%% The "real" document content comes below...

\title{The Channel Tunnel}
\author{Jake Humphrey}
%\date{} % Activate to display a given date or no date (if empty),
         % otherwise the current date is printed 

\begin{document}
\maketitle

\section{Introduction}

The concept of a man-made undersea tunnel connecting England and France has existed since the early 1800s, but it was not until 1986 that the construction effort resulting in the present Channel Tunnel was started.

Since its opening in 1994, the Channel Tunnel, affectionately nicknamed \emph{The Chunnel}, has transported over 300 million passengers and 300 million tonnes of freight between England and France across the English Channel.

As a work of engineering, the Tunnel is rather impressive. The only precedent of its type was the Seikan Tunnel in Japan. The Tunnel comprises two main rail tunnels and one service tunnel between them. The boring itself made use of a stratum of chalk marl, conducive to tunneling. All its have acheivements have contributed to its designation as one of the Seven Wonders of the Modern World by the American Society of Civil Engineers.

The Channel Tunnel offers both freight and passenger access, the latter being operated by Eurostar International Ltd, in addition to a roll-on roll-off shuttle service for road vehicles named Eurotunnel Le Shuttle.

This document seeks to give the reader, having a background in an engineering discipline, an overview of the history, construction, and operation of the Channel Tunnel, and an insight into its impact on transport infrastructure and the economy in England and France.

\section{History of Channel Crossing}
\subsection{Pre-tunnel}

The earliest proposal for connecting England and France beneath the Channel was made by  Albert Mathieu, a French mining engineer, in 1802, and included oil-lamp illumination and an artificial island midway across for changing the horses of one's cart.



\subsection{Construction of the Tunnel}
\section{Operation}
\section{Impact}
\subsection{England}
\subsection{France}
\section{Conclusion}

\end{document}
