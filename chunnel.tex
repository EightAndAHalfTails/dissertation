% !TEX TS-program = pdflatex
% !TEX encoding = UTF-8 Unicode

% This is a simple template for a LaTeX document using the "article" class.
% See "book", "report", "letter" for other types of document.

\documentclass[12pt]{article} % use larger type; default would be 10pt
\linespread{1.6} % `double' spacing

\usepackage[utf8]{inputenc} % set input encoding (not needed with XeLaTeX)

%%% Examples of Article customizations
% These packages are optional, depending whether you want the features they provide.
% See the LaTeX Companion or other references for full information.

%%% PAGE DIMENSIONS
\usepackage{geometry} % to change the page dimensions
\geometry{a4paper} % or letterpaper (US) or a5paper or....
% \geometry{margin=2in} % for example, change the margins to 2 inches all round
% \geometry{landscape} % set up the page for landscape
%   read geometry.pdf for detailed page layout information

\usepackage{graphicx} % support the \includegraphics command and options

% \usepackage[parfill]{parskip} % Activate to begin paragraphs with an empty line rather than an indent

%%% PACKAGES
\usepackage{booktabs} % for much better looking tables
\usepackage{array} % for better arrays (eg matrices) in maths
\usepackage{paralist} % very flexible & customisable lists (eg. enumerate/itemize, etc.)
\usepackage{verbatim} % adds environment for commenting out blocks of text & for better verbatim
\usepackage{subfig} % make it possible to include more than one captioned figure/table in a single float
% These packages are all incorporated in the memoir class to one degree or another...

%%% HEADERS & FOOTERS
\usepackage{fancyhdr} % This should be set AFTER setting up the page geometry
\pagestyle{fancy} % options: empty , plain , fancy
\renewcommand{\headrulewidth}{0pt} % customise the layout...
\lhead{}\chead{}\rhead{}
\lfoot{}\cfoot{\thepage}\rfoot{}

%%% SECTION TITLE APPEARANCE
%\usepackage{sectsty}
%\allsectionsfont{\sffamily\mdseries\upshape} % (See the fntguide.pdf for font help)
% (This matches ConTeXt defaults)

%%% ToC (table of contents) APPEARANCE
\usepackage[nottoc,notlof,notlot]{tocbibind} % Put the bibliography in the ToC
\usepackage[titles,subfigure]{tocloft} % Alter the style of the Table of Contents
\renewcommand{\cftsecfont}{\rmfamily\mdseries\upshape}
\renewcommand{\cftsecpagefont}{\rmfamily\mdseries\upshape} % No bold!

% Hyperlinks, URLs etc.
\usepackage{hyperref}
\usepackage{url}
\hypersetup{
    colorlinks=true,
    citecolor=black,
    urlcolor=black,
    linkcolor=black,
    pagecolor=black,
    anchorcolor=black
}

%%% END Article customizations

%%% The "real" document content comes below...

\title{The Channel Tunnel}
\author{Jake Humphrey}
\date{} % Activate to display a given date or no date (if empty),
         % otherwise the current date is printed 

\begin{document}
\maketitle

\section{Introduction}

The concept of a man-made undersea tunnel connecting England and France has existed since the early 1800s, but it was not until 1986 that the construction effort resulting in the present Channel Tunnel was started.

Since its opening in 1994, the Channel Tunnel, affectionately nicknamed \emph{The Chunnel}, has transported over 300 million passengers and 300 million tonnes of freight between England and France across the English Channel.

As a work of engineering, the Tunnel is rather impressive. The only precedent of its type was the Seikan Tunnel in Japan. The Channel Tunnel comprises two main rail tunnels and one service tunnel between them. The boring itself made use of a stratum of chalk marl, which has properties conducive to tunneling. All of its acheivements have contributed to its designation as one of the Seven Wonders of the Modern World by the American Society of Civil Engineers.

The Channel Tunnel offers both freight and passenger access, the latter being operated by Eurostar International Ltd, in addition to a roll-on roll-off shuttle service for road vehicles named Eurotunnel Le Shuttle.

This document seeks to give the reader (having a background in an engineering discipline) an overview of the history, construction, and operation of the Channel Tunnel, and an insight into its impact on transport infrastructure and the economy in England and France.

\section{History of Channel Crossing}
\subsection{Planning}

The earliest proposal for connecting England and France beneath the Channel was made by  Albert Mathieu, a French mining engineer, in 1802, and included oil-lamp illumination and an artificial island midway across for changing the horses of one's carriage.\cite{eurotunnel-build}

In 1834 eccentric French engineer and entrepreneur Aimé Thomé de Gamond proposed his first projects for a railway line beneath the English Channel. It was met with indifference from both English and French authorities, who at the time preferred to stay separated from their neighbours.

Gamond presented another proposal to the French Emperor Napoleon III in 1856 detailing a rail line from Cap Gris-Nez to Eastwater Point with a port\slash airshaft on the Varne sandbank. Gamond's preliminary surveying operations had estimated the cost at 170 million francs, or less than £7 million in the money of the time.\cite{ny-gamond}

Gamond proposed a total of seven designs over his lifetime. In 1867 his proposal was finally accepted by Napoleon III and Queen Victoria but was brought to an abrupt end by the Franco-Prussian War of 1870. Sadly, Gamond never saw his dream realised; he died ruined and humiliated in 1876\cite{gamond}

Ironically, this same year an official Anglo-French protocol was established for a cross-Channel railway tunnel\cite{bris}, and in 1881, the Anglo-French Submarine Railway Company conducted preliminary exploratory work on both the English and French sides. A couple of pilot tunnels no longer than 2km each had been dug when the project was abandoned in May 1882, over fears that the tunnel would compromise English national security.

The idea was next brought up nearly 40 years later, after the First World War, at the Paris Peace Conference in 1919, by British Prime Minister David Lloyd George. The suggestion was made as assurance that Britain was willing to defend France in the event of another German attack. However, the proposal was not taken seriously by the French and nothing ever came of it.

Another undeveloped proposal made in 1929 estimated the cost of construction to be about \$150. Military concerns of both nations had been addressed in the proposal, which included floodable sections of the tunnel to block access by either side. However, military leaders were not convinced. In addition, some English objected to the \emph{tourism} the project's completion would bring, which would supposedly ruin England's ``splendid isolation'' and ``make England a holiday resort for hordes of more or less undesirable people, who would introduce foreign customs, deface the countryside, and otherwise interrupt English habits of living''.\cite{pop29}

With air power gaining dominance in the military, the effect of a tunnel on national security became less and less significant. In 1955, British and French governements began to support technical and geological surveys. This culminated in a government-funded project to dig twin tunnels, designed to accommodate car shuttle wagons, on either side of a service tunnel. Construction began in 1974, but was cancelled by the British governement in January 1975 due to growing concerns over EEC membership and the national economy.

In 1981 British Prime Minister Margaret Thatcher and French President François Mitterand agreed to set up a group inviting private companies to put forward propositions. Over the next few years several projects were submitted including a 4.5km suspension bridge, holding a road encased in a tube, a drive-through tunnel, and the high-speed rail link that was ultimately selected and which exists today.

\subsection{Construction of the Tunnel}
\section{Operation}
\section{Impact}
\subsection{England}
\subsection{France}
\section{Conclusion}

\begin{thebibliography}{9}

\bibitem{eurotunnel-build}
	How the Channel Tunnel was Built --- Eurotunnel Le Shuttle\\
	\url{eurotunnel.com/build}\\
	Fetched 2015-06-20.

\bibitem{ny-gamond}
	\textit{The New York Times}. 7 August 1866.\\
	\url{http://query.nytimes.com/mem/archive-free/pdf?res=9A00EFD9133DE53BBC4F53DFBE66838D679FDE}\\
	Fetched 2015-06-20.

\bibitem{gamond}
	Aimé Thomé de Gamond on Wikipedia\\
	\url{en.wikipedia.org/wiki/Aim%C3%A9_Thom%C3%A9_de_Gamond}\\
	Fetched 2015-06-20.

\bibitem{bris}
	\textit{The Brisbane Courier}. 1 March 1876.\\
	\url{trove.nla.gov.au/ndp/del/article/1398039}\\
	Fetched 2015-06-20.

\bibitem{pop29}
	\emph{Popular Mechanics} May 1929, pp. 767--768\\
	\url{books.google.com/books?id=wN4DAAAAMBAJ&pg=PA767&dq=Popular+Science+1930+plane+%22Popular+Mechanics%22&hl=en&ei=fxBvTp7pAoyhtwfhqq33CQ&sa=X&oi=book_result&ct=result&resnum=8&ved=0CEQQ6AEwBzgU#v=onepage&q&f=true}\\
	Fetched 2015-06-20.

\end{thebibliography}

\end{document}
